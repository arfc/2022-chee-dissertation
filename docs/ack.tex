For this dissertation and all I have learnt from my years in graduate school, 
I am deeply grateful to many amazing professors and colleagues at the University of 
Illinois at Urbana-Champaign, as well as Argonne National Laboratory. 
I want to express my deepest gratitudes to my advisors, Dr. Kathryn Huff and Dr. 
Madicken Munk. 
I am grateful to Professor Huff for her unconditional support over the years, and for 
giving me so much freedom in pursuing my research as I found my footing in the early 
years of graduate school. 
I am especially grateful for her ability to inspire me to look at the bigger picture of 
my work, and holding me to a high standard in all aspects.
I am grateful to Dr. Munk for her guidance and encouragement in my final year 
of graduate school, as I wrote this dissertation. 
Her enthusiasm with discussing results, and her experience and insights challenged me to 
gain a deeper understanding, intuition, and appreciation for reactor physics.  
I would also like to thank the professors who served in my committee: Dr. Tomasz 
Kozlowski, Dr. James Stubbins, and Dr. Huy Trong Tran, for the knowledge and guidance 
they provided in assessing this work. 
I also wish to thank my Argonne internship advisors, Dr. Bo Feng and Dr. Adam Nelson, 
for their technical guidance and invaluable career advice. 

I am thankful for the friendship of my fellow ARFC 
groupmates who not only made great colleagues and collaborators, but also made 
grad school fun -- Sun Myung Park, Greg Westphal, Jin Whan Bae, Anshuman Chaube, 
Dr. Andrei Rykhlevkii, Sam Dotson, Nathan Ryan, Lu Kissinger, Amanda Bachmann, 
Nataly Panczyk, Olek Yardas, Luke Seifert, and Zoe Ritcher. 
Special thanks to Nathan Ryan and Nataly Panczyk for their excellent proofreading 
and comments on this work. 
I am also thankful to the wonderful NPRE, WiN, and GSAC communities at UIUC, in which 
I found great friendships with many students, and rapport with caring faculty and staff. 

Finally, I want to thank my family, Joy, Gerard, and Ben for their unwavering support 
and encouragement from day one. 

This work was carried out in the Advanced Reactors and Fuel Cycles Group (ARFC) at the 
Nuclear, Plasma, and Radiological Engineering (NPRE) Department at the University of 
Illinois at Urbana-Champaign (UIUC). 
This research was supported by the Nuclear Regulatory Commission Faculty Development 
Program (award NRC-HQ-84-14-G-0054 Program B) and the Department of Nuclear, Plasma, 
and Radiological Engineering.
This research is part of the Blue Waters sustained-petascale computing project, 
which is supported by the National Science Foundation (awards OCI-0725070 and 
ACI-1238993) the State of Illinois, and as of December, 2019, the National 
Geospatial-Intelligence Agency. 
Blue Waters is a joint effort of the University of Illinois at Urbana-Champaign and 
its National Center for Supercomputing Applications.
This research was also supported by the Argonne Leadership Computing Facility's 
Director's Discretionary Allocation Program. 