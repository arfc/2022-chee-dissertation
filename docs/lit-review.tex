\chapter{Literature Review}
\label{chap:lit-review}

This chapter provides a literature review of relevant past research efforts 
giving context to this dissertation. 
I begin this literature review with an overview of the \gls{FHR} concept, 
then go into detail about one specific \gls{FHR} design: the \gls{AHTR}, 
previous efforts and technical challenges of modeling the design, and a 
description of how these efforts led to the \gls{OECD} \gls{NEA}'s initiation 
of the \gls{FHR} benchmark.
Next, I outline additive manufacturing's history and describe the current 
research towards applying additive manufacturing to the fabrication of 
nuclear reactor components. 
I also review previous efforts towards nuclear reactor design optimization, 
describe how additive manufacturing of nuclear reactor components enables 
optimization for less constrained reactor geometries, and types of optimization 
methods that can be leveraged in this expanded design space.   
Finally, I give a background of evolutionary algorithms and detail a specific 
evolutionary algorithm: the genetic algorithm and how it works to conduct global 
optimization robustly.

\section{Fluoride-Salt-Cooled High-Temperature Reactor}
\label{sec:fhr}
To ensure continued global use and expansion of nuclear energy technology, in 
2001, the \gls{OECD} and nuclear energy experts initiated the \gls{GIF} 
\cite{gif_technology_2002}.
The \gls{GIF} aims to enhance the role of nuclear energy in our global energy 
ecosystem by coordinating global research and development to test the 
feasibility and performance of fourth generation nuclear systems, with the goal 
of making Gen IV reactors available for industrial deployment by 2030 
\cite{gif_technology_2002}.
