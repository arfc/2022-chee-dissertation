\chapter{Literature Review}
\label{chap:lit-review}

This chapter provides a literature review of relevant past research efforts 
that give context to this dissertation. 
I begin with an overview of the \gls{FHR} concept, 
then go into detail about one specific \gls{FHR} design: the \gls{AHTR}, 
previous efforts and technical challenges of modeling the design, and a 
description of how these efforts led to the \gls{OECD} \gls{NEA}'s initiation 
of the \gls{FHR} benchmark.
Next, I outline additive manufacturing's history and describe the current 
research towards applying additive manufacturing to the fabrication of 
nuclear reactor components. 
I also review previous efforts towards nuclear reactor design optimization, 
describe how additive manufacturing of nuclear reactor components enables 
optimization for less constrained reactor geometries, and types of optimization 
methods, such as the evolutionary algorithm, that can be leveraged to find 
optimal reactor designs in the expanded design space.   
Finally, I give a background of the evolutionary algorithms and detail a specific 
evolutionary algorithm: the genetic algorithm and how it works to conduct global 
optimization robustly.

\section{Fluoride-Salt-Cooled High-Temperature Reactor System}
\label{sec:fhr}
To ensure continued global use and expansion of nuclear energy technology, in 
2001, the \gls{OECD} and nuclear energy experts initiated the \gls{GIF} 
\cite{gif_technology_2002}.
The \gls{GIF} aims to enhance the role of nuclear energy in our global energy 
ecosystem by coordinating global research and development to test the 
feasibility and performance of fourth generation nuclear systems, with the goal 
of making Gen IV reactors available for industrial deployment by 2030 
\cite{gif_technology_2002}.
The \gls{GIF} selected six Generation IV systems for further research and 
development based on target goals in four areas: sustainability, 
economics, safety and reliability, and proliferation resistance and physical 
protection \cite{gif_technology_2002}. 
The systems are: \glspl{GFR}, \glspl{LFR}, \glspl{MSR}, \glspl{SFR}, \glspl{SCWR}, 
and \glspl{VHTR} \cite{gif_technology_2002}. 
The \gls{FHR} concept introduced in 2003 uses a low-pressure liquid fluoride-salt 
coolant and high-temperature coated-particle \gls{TRISO} fuel, combining 
the best aspects of the \gls{MSR} and \gls{VHTR} systems respectively
\cite{forsberg_molten-salt-cooled_2003,facilitators_fluoride-salt-cooled_2013}.

\gls{MSR} systems produce fission power in a circulating molten salt fuel 
mixture. 
Researchers recommend molten fluoride salts because they have high uranium 
solubility, chemical stability, very low vapor pressure even at high 
temperatures, good heat transfer properties, resistance against radiation 
damage, and are inert to common structural materials 
\cite{rosenthal_molten-salt_1970}. 
Molten salt reactor coolants also introduce inherent safety due to the 
low system vapor pressure and the salts' high boiling temperature and 
volumetric heat capacity \cite{ho_molten_2013}.
Fluoride salt used in \glspl{FHR} is Li$_2$BeF$_4$ (FLiBe), 
which remains liquid without pressurization up to 1400 $^{\circ}$C and has a greater 
heat capacity than water \cite{ho_molten_2013,forsberg_fluoride-salt-cooled_2012}.
\gls{VHTR} systems use a once-through uranium cycle and leverage their 
high outlet temperature for high-temperature heat applications, such as 
hydrogen production. 
Graphite-moderated and helium-cooled, \glspl{VHTR} use \gls{TRISO} fuel
which withstands high burnup and temperature, enabling higher operating 
temperatures \cite{gif_technology_2002}.  
The advantages of higher operating temperatures include: increased power 
conversion efficiency, reduced waste heat generation, and co-generation and 
process heat capabilities \cite{scarlat_design_2014}.
However, the \glspl{VHTR} system's helium coolant is at 100 atm requiring a 
thick concrete vessel. 

By combining the FLiBE coolant from \gls{MSR} technology and 
\gls{TRISO} fuel from \gls{VHTR} technology, the \gls{FHR} benefits from 
a low operating pressure and large thermal margin enabled by the molten 
salt coolant and the thermal resilience of \gls{TRISO} particle fuel. 
Molten salt coolant's superior cooling properties compared to the \gls{VHTR}'s
helium coolant increases system safety with an atmospheric operating pressure 
instead of 100 atm. 
\gls{TRISO} solid fuel cladding in the \gls{FHR} system adds an extra barrier 
to fission product release compared to \glspl{MSR} with liquid fuel 
\cite{ho_molten_2013}.

Several types of \gls{FHR} conceptual designs exist worldwide: the \gls{PBFHR} 
developed at \gls{UCB} with circulating pebble-fuel 
\cite{scarlat_current_2014,krumwiede_three-dimensional_2013}, the \gls{SF-TMSR} 
developed at the \gls{SINAP} in China with static pebble-fuel \cite{liu_preliminary_2016}, 
the large central-station \gls{AHTR} at \gls{ORNL} \cite{holcomb_core_2011, varma_ahtr_2012} and 
the \gls{SmAHTR} at ORNL \cite{greene_pre-conceptual_2010} with static, plate fuel. 

\subsection{\acrlong{AHTR} Design}
This dissertation focuses on the prismatic \gls{FHR} design with hexagonal fuel 
assemblies consisting of \gls{TRISO} fuel particles embedded in planks, i.e., 
the \gls{AHTR} design developed by ORNL. 
The \gls{AHTR} has 3400 MWt thermal power and 1400 MW electric power with
inlet/outlet temperatures of 650/700$^{\circ}$C \cite{varma_ahtr_2012}.  
Figure \ref{fig:ahtr} shows the prismatic AHTR's fuel assembly and core 
configuration.  
\begin{figure}[]
    \centering
    \includegraphics[width=\linewidth]{ahtr.png} 
    \caption{\acrlong{AHTR} fuel assembly (left) and core configuration (right) 
    reproduced from \cite{ramey_monte_2018}.}
    \label{fig:ahtr}
\end{figure}
Each hexagonal fuel assembly features plate-type fuel consisting of eighteen 
planks arranged in three diamond-shaped sectors, with a central Y-shaped 
structure and external channel (wrapper).
The fuel planks contain an isostatically pressed carbon with fuel stripes 
on each plank's outer side.
Within each fuel stripe is a graphite matrix filled with \gls{TRISO} particles. 
The core consists of 252 assemblies radially surrounded by reflectors
\cite{ramey_monte_2018}. 
Chapter \ref{chap:fhr-benchmark} details the specifications of the AHTR geometry
modeled in this proposed work.

\subsection{Previous AHTR modeling efforts and challenges} 
\label{sec:previous_ahtr}