\chapter{Literature Review}
\label{chap:lit-review}

This chapter provides a literature review of relevant past research efforts 
that give context to this dissertation. 
I begin with an overview of the \gls{FHR} concept, 
then go into detail about one specific \gls{FHR} design: the \gls{AHTR}, 
previous efforts and technical challenges of modeling the design, and a 
description of how these efforts led to the \gls{OECD} \gls{NEA}'s initiation 
of the \gls{FHR} benchmark.
Next, I outline additive manufacturing's history and describe the current 
research towards applying additive manufacturing to the fabrication of 
nuclear reactor components. 
I also review previous efforts towards nuclear reactor design optimization, 
describe how additive manufacturing of nuclear reactor components enables 
optimization for less constrained reactor geometries, and types of optimization 
methods, such as the evolutionary algorithm, that can be leveraged to find 
optimal reactor designs in the expanded design space.   
Finally, I give a background of the evolutionary algorithms and detail a specific 
evolutionary algorithm: the genetic algorithm and how it works to conduct global 
optimization robustly.

\section{Fluoride-Salt-Cooled High-Temperature Reactor System}
\label{sec:fhr}
To ensure continued global use and expansion of nuclear energy technology, in 
2001, the \gls{OECD} and nuclear energy experts initiated the \gls{GIF} 
\cite{gif_technology_2002}.
The \gls{GIF} aims to enhance the role of nuclear energy in our global energy 
ecosystem by coordinating global research and development to test the 
feasibility and performance of fourth generation nuclear systems, with the goal 
of making Gen IV reactors available for industrial deployment by 2030 
\cite{gif_technology_2002}.
The \gls{GIF} selected six Generation IV systems for further research and 
development based on target goals in four areas: sustainability, 
economics, safety and reliability, and proliferation resistance and physical 
protection \cite{gif_technology_2002}. 
The systems are: \glspl{GFR}, \glspl{LFR}, \glspl{MSR}, \glspl{SFR}, \glspl{SCWR}, 
and \glspl{VHTR} \cite{gif_technology_2002}. 
The \gls{FHR} concept introduced in 2003 uses a low-pressure liquid fluoride-salt 
coolant and high-temperature coated-particle \gls{TRISO} fuel, combining 
the best aspects of the \gls{MSR} and \gls{VHTR} systems respectively
\cite{forsberg_molten-salt-cooled_2003,facilitators_fluoride-salt-cooled_2013}.

\gls{MSR} systems produce fission power in a circulating molten salt fuel 
mixture. 
Researchers recommend molten fluoride salts because they have high uranium 
solubility, chemical stability, very low vapor pressure even at high 
temperatures, good heat transfer properties, resistance against radiation 
damage, and are inert to common structural materials 
\cite{rosenthal_molten-salt_1970}. 
Molten salt reactor coolants also introduce inherent safety due to the 
low system vapor pressure and the salts' high boiling temperature and 
volumetric heat capacity \cite{ho_molten_2013}.
Fluoride salt used in \glspl{FHR} is Li$_2$BeF$_4$ (FLiBe), 
which remains liquid without pressurization up to 1400 $^{\circ}$C and has a greater 
heat capacity than water \cite{ho_molten_2013,forsberg_fluoride-salt-cooled_2012}.
\gls{VHTR} systems use a once-through uranium cycle and leverage their 
high outlet temperature for high-temperature heat applications, such as 
hydrogen production. 
Graphite-moderated and helium-cooled, \glspl{VHTR} use \gls{TRISO} fuel
which withstands high burnup and temperature, enabling higher operating 
temperatures \cite{gif_technology_2002}.  
The advantages of higher operating temperatures include: increased power 
conversion efficiency, reduced waste heat generation, and co-generation and 
process heat capabilities \cite{scarlat_design_2014}.
However, the \glspl{VHTR} system's helium coolant is at 100 atm requiring a 
thick concrete vessel. 

By combining the FLiBE coolant from \gls{MSR} technology and 
\gls{TRISO} fuel from \gls{VHTR} technology, the \gls{FHR} benefits from 
a low operating pressure and large thermal margin enabled by the molten 
salt coolant and the thermal resilience of \gls{TRISO} particle fuel. 
Molten salt coolant's superior cooling properties compared to the \gls{VHTR}'s
helium coolant increases system safety with an atmospheric operating pressure 
instead of 100 atm. 
\gls{TRISO} solid fuel cladding in the \gls{FHR} system adds an extra barrier 
to fission product release compared to \glspl{MSR} with liquid fuel 
\cite{ho_molten_2013}.

Several types of \gls{FHR} conceptual designs exist worldwide: the \gls{PBFHR} 
developed at \gls{UCB} with circulating pebble-fuel 
\cite{scarlat_current_2014,krumwiede_three-dimensional_2013}, the \gls{SF-TMSR} 
developed at the \gls{SINAP} in China with static pebble-fuel \cite{liu_preliminary_2016}, 
the large central-station \gls{AHTR} at \gls{ORNL} \cite{holcomb_core_2011, varma_ahtr_2012} and 
the \gls{SmAHTR} at ORNL \cite{greene_pre-conceptual_2010} with static, plate fuel. 

\subsection{\acrlong{AHTR} Design}
This dissertation focuses on the prismatic \gls{FHR} design with hexagonal fuel 
assemblies consisting of \gls{TRISO} fuel particles embedded in planks, i.e., 
the \gls{AHTR} design developed by ORNL. 
The \gls{AHTR} has 3400 MWt thermal power and 1400 MW electric power with
inlet/outlet temperatures of 650/700$^{\circ}$C \cite{varma_ahtr_2012}.  
Figure \ref{fig:ahtr} shows the prismatic AHTR's fuel assembly and core 
configuration.  
\begin{figure}[]
    \centering
    \includegraphics[width=\linewidth]{ahtr.png} 
    \caption{\acrlong{AHTR} fuel assembly (left) and core configuration (right) 
    reproduced from \cite{ramey_monte_2018}.}
    \label{fig:ahtr}
\end{figure}
Each hexagonal fuel assembly features plate-type fuel consisting of eighteen 
planks arranged in three diamond-shaped sectors, with a central Y-shaped 
structure and external channel (wrapper).
The fuel planks contain an isostatically pressed carbon with fuel stripes 
on each plank's outer side.
Within each fuel stripe is a graphite matrix filled with \gls{TRISO} particles. 
The core consists of 252 assemblies radially surrounded by reflectors
\cite{ramey_monte_2018}. 
Chapter \ref{chap:fhr-benchmark} details the specifications of the AHTR geometry
modeled in this proposed work.

\subsection{Previous AHTR modeling efforts and challenges} 
\label{sec:previous_ahtr}
The \gls{AHTR} core design differs significantly from the present \gls{LWR} 
systems' cores. 
These differences lead to modeling challenges with the current tools and also 
highlight the need for verification and validation of current simulation tools 
for \gls{AHTR} physics \cite{ramey_monte_2018}. 
Verification and validation of \gls{AHTR} neutronics and thermal-hydraulics 
simulation capabilities supports the \gls{AHTR} design's licensure and 
eventual goal of \gls{AHTR} deployment 
\cite{rahnema_phenomena_2019,rahnema_current_2015}.
In this section, I outline the previous efforts taken to model and validate 
the \gls{AHTR}'s neutronics and thermal-hydraulics. 

\subsubsection{AHTR Neutronics Modeling}
Several neutronics studies conducted along the way to the current \gls{AHTR} 
design have shed light on the technical challenges facing the design 
\cite{ramey_monte_2018,holcomb_fluoride_2013,greene_pre-conceptual_2010}. 
\gls{Georgia Tech} led an Integrated Research Project to 
understand challenges in \gls{AHTR} materials and modeling its neutronics and 
thermal-hydraulics \cite{zhang_integrated_2019}. 
During the research project, a panel of subject matter experts 
generated a \gls{PIRT}.
The \gls{PIRT} identifies areas that need additional research to better 
understand important phenomena for adequate future modeling
\cite{rahnema_phenomena_2019}. 
Table \ref{tab:phenomena} lists the phenomena identified as requiring further 
research. 
\begin{table}[]
    \centering
    \onehalfspacing
    \caption{\acrlong{PIRT} identified \acrlong{AHTR} physical phenomena requiring 
    further research \cite{rahnema_phenomena_2019}.}
	\label{tab:phenomena}
    \footnotesize
    \begin{tabular}{l|l}
    \hline
    \textbf{Category} & \textbf{Phenomena} \\ \hline
    Fundamental cross section data & - Moderation in FliBe \\
    & - Thermalization in FliBe \\
    & - Absorption in FliBe \\
    & - Thermalization in carbon \\
    & - Absorption in carbon \\ \hline
    Material Composition & - Fuel particle distribution \\ \hline
    Computational Methodology & - Solution Convergence \\ 
    & - Granularity of depletion regions \\
    & - Multiple heterogeneity treatment for generating multigroup \\ 
    & cross sections \\
    & - Selection of multigroup structure \\
    & - Boundary conditions for multigroup cross section generation \\ \hline 
    General Depletion & - Spectral history \\ \hline 
    \end{tabular}
\end{table}

The \emph{triple heterogeneous} \gls{AHTR} fuel, comprised of \gls{TRISO} 
particles embedded in strategically arranged plates, presents simulation 
challenges. 
Researchers must obtain detailed reference power distributions with individual 
\gls{TRISO} particle fidelity to best understand nuances in the physics, such 
as self-shielding.
Deterministic codes that use multigroup cross sections and traditional 
homogenization methods \cite{ramey_monte_2018}, insufficiently capture the 
correct physics in \glspl{AHTR} due to these multiple heterogeneities
\cite{ramey_monte_2018}. 
In the \gls{AHTR}, single and multiple slab homogenization decreased total 
neutron transport simulation time by an order of 10; however, the homogenization 
introduced a nontrivial $k_{eff}$ error of $\sim$3\% 
\cite{ramey_monte_2018,cisneros_neutronics_2012}.
To determine the feasibility and safety of the \gls{AHTR} design, researchers 
must calculate core physics parameters to an acceptable uncertainty. 
With Monte Carlo neutron transport, increasing neutron histories reduces statistical 
uncertainty but increases computational cost typically, requiring the use of 
supercomputers to run the simulations.

This \gls{AHTR} presents another technical challenge: the uncertainty of 
graphite moderator material properties: densities, temperatures, and thermal 
scattering data.
Problematically, the thermal scattering data ($S(\alpha,\beta)$ matrices) for 
the bound nuclei in \gls{FLiBe} salts are lacking \cite{ramey_monte_2018}. 
Mei et al. \cite{mei_investigation_2013} and Zhu et al. \cite{zhu_thermal_2017} 
examined the thermal scattering behavior of solid and liquid \gls{FLiBe}.
They concluded that the bound and free atom cross section of \gls{FLiBe} are 
identical above 0.1eV and diverges below 0.01eV, which means that the use or 
absence of thermal scattering data will impact the accuracy of the results 
\cite{ramey_monte_2018}. 

\subsubsection{AHTR Multiphysics Modeling}
In past effort towards multiphysics modeling of the \gls{AHTR}, Gentry et al 
\cite{gentry_development_2016} developed an adapted lattice physics-to-core 
simulator two-step procedure with Serpent \cite{leppanen_serpent_2014} 
and \gls{NESTLE} \cite{turinsky_nestle_1994} for the \gls{AHTR} design. 
The adapted lattice physics-to-core simulator two-step procedure proved to be 
successful for \glspl{LWR} in which few group assembly homogenized group 
constants are generated by 2-D transport lattice calculation and then core 
analysis is performed by 3-D nodal simulation 
\cite{koebke_new_1980,gentry_development_2016}.
\gls{NESTLE}'s thermal-hydraulics utilizes a \gls{HEM} model for two-phase 
flow and it solves the few-group neutron diffusion equation utilizing the
\gls{NEM} for cartesian and hexagonal reactor geometries.  
Gentry et al concluded that the method required further accuracy improvements 
by improving reflector model and optimizing coarse energy group structure further.
Lin \cite{lin_thermal_2020} used RELAP5, a system-level code, to perform 
\gls{AHTR} thermal hydraulics transient simulations to investigate the 
capability of the passive heat removal system. 
In this \gls{AHTR} RELAP5 model, the 252 assemblies are separated into four 
concentric rings and a uniform power distribution is assigned to the fuel 
assemblies in each ring, and more fidelity is placed on the primary and 
\gls{DRACS} system loops. 
RELAP5 is a system-level code, Lin utilized it in transient scenarios to determine 
the temperature at various locations in the system loop by assuming a power 
value for fuel assemblies in each ring \cite{lin_thermal_2020}. 
However, this method is not ideal for transient scenarios with tightly coupled 
neutronics and thermal-hydraulics. 

\subsection{FHR Benchmark}
The previous section highlights the singular efforts to model 
different aspects of the \gls{AHTR}'s neutronics and thermal hydraulics, with
each author describing their modeling difficulties. 
However, there lacked a robust and methodical method for evaluating the 
simulation software and comparing the results generated by individual 
researchers.

To gain a wholistic view of the \gls{AHTR}'s modeling challenges and 
cross-verify available \gls{AHTR} modeling tools, in 2019, the 
\gls{OECD}-\gls{NEA} initiated the \gls{FHR} benchmarking exercise 
of the \gls{AHTR} design \cite{noauthor_fluoride_nodate}.
Several organizations participate in the benchmark with various Monte Carlo
and Deterministic neutronics software, such as Serpent \cite{leppanen_serpent_2014}, 
OpenMC \cite{romano_openmc_2013}, and WIMS \cite{lindley_current_2017}. 

The benchmark will have three phases: a single fuel assembly simulation 
without burnup (Phase I), full core depletion (Phase II), and multi-physics 
feedback (Phase III). 
The benchmark aims to identify the applicability, accuracy, and practicality 
of the latest methods and codes to assess the current state 
of the art FHR simulation and modeling \cite{petrovic_preliminary_2021}. 
The benchmark also enables the cross-verification of software and methods 
for the challenging \gls{AHTR} geometry, which is especially useful since 
applicable reactor physics experiments for code validation are scarce 
\cite{petrovic_fhrahtr_2019,petrovic_preliminary_2021}. 
Chapter \ref{chap:fhr-benchmark} will provide a detailed description of the 
benchmark phases and results obtained so far.