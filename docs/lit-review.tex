\chapter{Literature Review}
\label{chap:lit-review}

This chapter provides a literature review of relevant past research efforts 
giving context to this dissertation. 
I begin this literature review with an overview of the \gls{FHR} concept, 
then go into detail about one specific \gls{FHR} design: the \gls{AHTR}, 
previous efforts and technical challenges of modeling the design, and a 
description of how these efforts led to the \gls{OECD} \gls{NEA}'s initiation 
of the \gls{FHR} benchmark.
Next, I outline additive manufacturing's history and describe the current 
research towards applying additive manufacturing to the fabrication of 
nuclear reactor components. 
I also review previous efforts towards nuclear reactor design optimization, 
describe how additive manufacturing of nuclear reactor components enables 
optimization for less constrained reactor geometries, and types of optimization 
methods that can be leveraged in this expanded design space.   
Finally, I give a background of evolutionary algorithms and detail a specific 
evolutionary algorithm: the genetic algorithm and how it works to conduct global 
optimization robustly.

\section{Fluoride-Salt-Cooled High-Temperature Reactor System}
\label{sec:fhr}
To ensure continued global use and expansion of nuclear energy technology, in 
2001, the \gls{OECD} and nuclear energy experts initiated the \gls{GIF} 
\cite{gif_technology_2002}.
The \gls{GIF} aims to enhance the role of nuclear energy in our global energy 
ecosystem by coordinating global research and development to test the 
feasibility and performance of fourth generation nuclear systems, with the goal 
of making Gen IV reactors available for industrial deployment by 2030 
\cite{gif_technology_2002}.
The \gls{GIF} selected six Generation IV systems for further research and 
development based on target goals in four areas: sustainability, 
economics, safety and reliability, and proliferation resistance and physical 
protection \cite{gif_technology_2002}. 
The systems are: \glspl{GFR}, \glspl{LFR}, \glspl{MSR}, \glspl{SFR}, \glspl{SCWR}, 
and \glspl{VHTR} \cite{gif_technology_2002}. 

The \gls{FHR} concept introduced in 2003 uses a low-pressure liquid fluoride-salt 
coolant and high-temperature coated-particle fuel, combining 
the best aspects of the \gls{MSR} and \gls{VHTR} systems respectively
\cite{forsberg_molten-salt-cooled_2003,facilitators_fluoride-salt-cooled_2013}.
High-temperature performance and overall chemical stability make molten 
fluoride salts desirable as working fluids for nuclear reactors
\cite{scarlat_design_2014}.  
Molten salt reactor coolant also introduces inherent safety due to the 
salts' high boiling temperature and high volumetric heat capacity
\cite{ho_molten_2013}.
Fluoride salt used in \glspl{FHR} is Li$_2$BeF$_4$ (FLiBe), 
which remains liquid without pressurization up to 1400 $^{\circ}$C and has a greater 
heat capacity than water \cite{ho_molten_2013,forsberg_fluoride-salt-cooled_2012}. 