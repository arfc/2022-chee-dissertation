\chapter{AHTR Modeling and Optimization Methodology}
In this chapter, I describe the modeling and optimization methodology of
ROLLO \gls{AHTR} optimization for non-conventional geometries and parameters.
To wholly explore the design space enabled by additive manufacturing, the 
optimization tool should enable placement of fuel, moderation, and coolant material 
in any possible location, within physical limits. 
Since exploration of non-conventional geometries and parameters has barely been
attempted (previous attempts described in Section \ref{sec:lit-review-reactor-arbitrary}), 
this dissertation attempts a first go at beginning to explore the large design space.  
The work done for this dissertation is only an intermediate step towards developing 
a truly arbitrary geometry expression. 

In the subsequent sections, I will define the optimization problem, describe the 
AHTR geometries . 
Then, I will describe the software used to model them and the specific models. 

\section{Optimization Problem Definition}
In an effort towards optimizing reactor design for non-conventional geometries 
and parameters.
I chose to vary the following \gls{AHTR} parameters: 
\begin{itemize}
    \item \gls{TRISO} particle packing fraction distribution, 
    $\rho_{TRISO}(\vec{r})$
    \item Total fuel packing fraction
    \item \gls{FLiBe} coolant channel shape 
\end{itemize} 
The TRISO packing fraction distribution enables exploration of how heterogenous 
fuel distributions impact reactor performance.  