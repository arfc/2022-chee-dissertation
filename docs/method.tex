\chapter{AHTR Modeling and Optimization Methodology}
In this chapter, I describe the modeling and optimization methodology of
\gls{ROLLO} \gls{AHTR} optimization for non-conventional geometries and parameters.
To wholly explore the design space enabled by additive manufacturing, the 
optimization tool should enable placement of fuel, moderation, and coolant material 
in any possible location, within physical limits. 
Since exploration of non-conventional geometries and parameters has barely been
attempted (previous attempts described in Section \ref{sec:lit-review-reactor-arbitrary}), 
this dissertation attempts a first go at beginning to explore the large design space.  
The work done for this dissertation is only an intermediate step towards developing 
a truly arbitrary geometry expression. 

%In the subsequent sections, I will define the optimization problem, describe the 
%AHTR geometries . 
%Then, I will describe the software used to model them and the specific models. 

\section{Optimization Problem Definition}
In an effort towards optimizing reactor design for non-conventional geometries 
and parameters.
I chose to vary the following \gls{AHTR} parameters: 
\begin{itemize}
    \item \gls{TRISO} particle packing fraction distribution, 
    $\rho_{TRISO}(\vec{r})$
    \item Total fuel packing fraction
    \item \gls{FLiBe} coolant channel shape 
\end{itemize} 
The TRISO packing fraction distribution variation enables exploration of how 
heterogenous fuel distributions impact reactor performance;  
while the FliBe coolant channel shape variation enables exploration of how non-uniform 
channel shapes impact reactor performance. 

I selected three key \gls{AHTR} optimization objectives that address contrasting reactor 
core qualities. 
Table \ref{tab:objectives} describes each objective, how I quantified them, and the motivation.
\begin{table}[]
    \centering
    \onehalfspacing
    \caption{\acrfull{ROLLO} optimization problem objectives with their quantification 
    descriptions, and motivation.}
	\label{tab:objectives}
    \footnotesize
    \begin{tabular}{p{4cm}p{5cm}p{5cm}}
    \hline 
    \textbf{Objective}& \textbf{Quantification}& \textbf{Motivation} \\
    \hline
    Minimize fuel amount & Minimize total fuel packing fraction & Cost savings, Non-proliferation \\ 
    \hline
    Maximize heat transfer & Minimize maximum temperature & Enable system to perform at a higher power with minimized thermal stress \\
    \hline
    Minimize power peaking & Minimize power peaking factor normalized by fuel distribution & Efficient fuel utilization, longer core life, safety\\
    \hline
    \end{tabular}
\end{table}
I will be applying optimization process to the \gls{AHTR} slab and \gls{AHTR} one-third
assembly geometries.
The slab optimization acts as a preliminary study to inform the more complex \gls{AHTR} one-third
assembly optimization setup. 
In the next section, I will describe both geometries. 

\section{AHTR Geometry}
The optimization process is applied to both the \gls{AHTR} slab and \gls{AHTR} one-third
assembly geometries.

\subsection{AHTR Slab Geometry}

