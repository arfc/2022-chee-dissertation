\chapter{Introduction}
\label{sec:intro}

Increased global surface temperatures, sea levels, and severe weather events 
caused by elevated \gls{GHG} concentrations show the negative impact of 
climate change on natural and human systems \cite{noauthor_climate_2018}. 
Energy use and production contribute two-thirds of total \gls{GHG}
emissions \cite{noauthor_climate_2018}.
Furthermore, as the human population increases and previously underdeveloped 
nations rapidly industrialize, global energy demand will continue to rise.   
Energy generation technology profoundly impacts climate change. 
Large-scale emissions-free nuclear power deployment could significantly reduce 
\gls{GHG} production but faces both cost and perceived adverse safety challenges 
\cite{noauthor_climate_2018, petti_future_2018}. 
To provide low-carbon electricity worldwide, the nuclear power industry must 
overcome cost and safety challenges to ensure continued global use and 
expansion of nuclear energy technology.

Wide-spread adoption of additive manufacturing methods in the nuclear industry 
could drastically reduce reactor fabrication costs and deployment timelines 
and improve reactor safety. 
These reductions are achieved by combining multiple systems and assembled 
components into single parts, tailoring local material properties, and 
enabling geometry redesign for increased safety and performance 
\cite{simpson_considerations_2019}. 
Additive manufacturing (i.e., 3D printing) has progressed rapidly in the last 
30 years, from rapid design prototyping with polymers in the automotive industry 
to metal component-scale production. With further advancement of additive 
manufacturing technologies, a reactor core could be 3D printed within the next decade. 
\gls{ORNL} leads this initiative through the 2019 \gls{TCR} Demonstration Program. 
The \gls{TCR} program will 3D print a microreactor by leveraging recent scientific 
achievements in additive manufacturing, nuclear materials, machine learning, and 
computational modeling and simulation \cite{terrani_transformational_2019}.

Additive manufacturing of reactor core components remove the geometric constraints of 
conventional fuel manufacturing. 
Reactor designers are no longer limited by traditional geometric shapes that are easy 
to manufacture with traditional processes, such as slabs as fuel planks, cylinders as 
fuel rods, spheres as fuel pebbles, and axis-aligned coolant channels 
\cite{sobes_artificial_2020}. 
Fabricating reactor core component using additive manufacturing can further optimize 
and improve core geometries to enhance reactor performance and safety at lower costs 
\cite{bergeron_early_2018}.