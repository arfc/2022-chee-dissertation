\chapter{Introduction}
\glsresetall
\label{sec:intro}

Access to electricity plays a vital role in improving standards of living, 
education, and health around the world \cite{petti_future_2018}. 
As the human population increases and previously underdeveloped 
nations rapidly industrialize, global energy demand will continue to rise with 
worldwide electricity use expected to grow $45\%$ by 2040 
\cite{petti_future_2018,noauthor_us_nodate}.
Expanding access to energy while at the same time drastically reducing 
greenhouse gas emissions that cause global warming and climate challenge 
is among the central challenges confronting humankind in the 21st century
\cite{petti_future_2018}. 

The latest scientific report by the \gls{IPCC} finds changes in the Earth's climate 
in every region and across the whole climate system. 
Increased global surface temperatures, sea levels, and severe weather events 
caused by elevated \gls{GHG} concentrations show the negative impact of 
climate change on natural and human systems \cite{noauthor_climate_2018}. 
Concerted international efforts over the past 20 years have increased the amount 
of electricity generated by wind, solar and other renewable sources, but have 
failed to displace fossil fuels from the mix. 
As a matter of fact, in 2017, fossil fuels produced more electricity - in relative 
and absolute terms - than ever before \cite{noauthor_nuclear_nodate}.
All technologies that can contribute towards solving one of the greatest challenges 
faced by humankind should be deployed.

Nuclear power plants produce no greenhouse gas emissions during operation, and over 
the course of its life-cycle, nuclear produces about the same amount of carbon 
dioxide-equivalent emissions per unit of electricity as wind, and one-third of 
the emissions per unit of electricity when compared with solar 
\cite{noauthor_nuclear_nodate-1}.
Nuclear fission power is a mature technology that provides essential base load 
power that can complement daily and seasonal peaking of renewables; many 
projections of energy portfolios in carbon-limited scenarios include fission power as
an important contributor to deep decarbonization of the electricity sector in tandem 
with energy diversification objectives \cite{petti_future_2018,
noauthor_nuclear_nodate-1}.
Large-scale emissions-free nuclear power deployment could significantly 
reduce \gls{GHG} production but faces both cost and perceived adverse safety 
challenges \cite{noauthor_climate_2018, petti_future_2018}. 
To provide low-carbon electricity worldwide, the nuclear power industry must 
overcome cost and safety challenges to ensure continued global use and 
expansion of nuclear energy technology.

Wide-spread adoption of additive manufacturing methods in the nuclear industry 
could drastically reduce reactor fabrication costs and deployment timelines 
and improve reactor safety \cite{simpson_considerations_2019}. 
These reductions are achieved by combining multiple systems and assembled 
components into single parts, tailoring local material properties, and 
enabling geometry redesign for increased safety and performance 
\cite{simpson_considerations_2019}. 
Additive manufacturing (i.e., 3D printing) has progressed rapidly in the last 
30 years, from rapid design prototyping with polymers in the automotive industry 
to metal component-scale production. With further advancement of additive 
manufacturing technologies, a reactor core could be 3D printed within the next decade. 
\gls{ORNL} leads this initiative through the 2019 \gls{TCR} Demonstration Program. 
The \gls{TCR} program will 3D print a microreactor by leveraging recent scientific 
achievements in additive manufacturing, nuclear materials, machine learning, and 
computational modeling and simulation \cite{terrani_transformational_2019}.

Additive manufacturing of reactor core components remove the geometric constraints of 
conventional fuel manufacturing. 
Reactor designers are no longer limited by traditional geometric shapes that are easy 
to manufacture with traditional processes, such as slabs as fuel planks, cylinders as 
fuel rods, spheres as fuel pebbles, and axis-aligned coolant channels 
\cite{sobes_artificial_2020}. 
Fabricating reactor core component using additive manufacturing can further optimize 
and improve core geometries to enhance reactor performance and safety at lower costs 
\cite{bergeron_early_2018}.

Reactor optimization for arbitrary geometries enabled by additive manufacturing is a 
new concept, and few research demonstrations have been done to explore the large 
new design space. 
Previous efforts include Sobes et al. \cite{sobes_artificial_2020} using genetic 
algorithms to find minimum volume geometric configurations for the TCR reactor. 
This study represented arbitrary geometry variations using right cylinders. 
Also, See et al. \cite{see_design_2022} optimized the TCR's outlet plenum design 
by varying inlet channel positioning and outer wall shape.  
The expanded design space associated with an arbitrary reactor geometry increases 
the time for reactor designers to thoroughly explore and find optimal geometries. 
Instead, we can leverage \gls{AI} optimization methods, such as evolutionary algorithms, 
to rapidly explore the large design space to find global optimal designs. 
\gls{AI} does not replace the human reactor designer but shifts the human designer's 
focus away from conjecturing suitable geometries to defining design criteria to 
find optimal designs \cite{sobes_artificial_2020}. 
Thus, when the human designer changes the reactor criteria, the AI model will quickly 
adapt and produce new global optimal designs to fit the new criteria.

Thoroughly exploring the design space enabled by additive manufacturing should allow 
the placement of fuel, moderation, and coolant material in any possible location 
within physical limits. 
In this dissertation, I explore the large design space while acknowledging that 
this work is only an intermediate step towards developing a truly arbitrary 
geometry expression. 
I apply evolutionary algorithm optimization to vary the fuel distribution and 
coolant channel shape in a \gls{FHR}, to minimize three reactor key performance 
metrics: total fuel amount, maximum temperature, and fuel-normalized power peaking 
factor. 
For this dissertation, I designed the \gls{ROLLO} tool \cite{chee_rollo_2021} to drive 
the evolutionary algorithm optimization process, and used OpenMC 
\cite{romano_openmc:_2015} and Moltres \cite{lindsay_introduction_2018} software to 
model the reactor's neutronics and thermal-hydraulics, respectively.  

In 2000, the \gls{DOE} initiated the Generation IV International Forum, a collective 
of countries committed to the joint development of the next generation of fission 
reactors.
They aim to enhance the role of nuclear energy in our global energy ecosystem by 
leading and planning research and development to support a new and innovative 
Generation IV nuclear energy systems \cite{gif_technology_2002}.
Generation IV nuclear systems target goals in four areas: sustainability, 
economics, safety and reliability, and proliferation resistance and physical 
protection \cite{gif_technology_2002}. 
The Generation IV International Forum's methodology working groups developed 
an evaluation and selection methodology based on these goals 
and correspondingly selected six Generation IV systems: \glspl{GFR}, 
\glspl{LFR}, \glspl{MSR}, \glspl{SFR}, \glspl{SCWR}, and \glspl{VHTR} 
\cite{gif_technology_2002}. 

In this dissertation, I apply optimization methods to the \gls{FHR} concept, which 
combines the best aspects of \gls{MSR} and \gls{VHTR} technologies. 
\glspl{FHR} use high-temperature coated-particle fuel (similar to the \glspl{VHTR}) 
and a low-pressure liquid fluoride-salt coolant (similar to the \glspl{MSR})
\cite{forsberg_fluoride-salt-cooled_2012,facilitators_fluoride-salt-cooled_2013}.
This dissertation focuses on a prismatic \gls{FHR} design with hexagonal fuel 
assemblies consisting of \gls{TRISO} fuel particles embedded in planks, i.e., 
the \gls{AHTR} design.
The \gls{AHTR}'s fuel geometry has \emph{triple heterogeneity} resulting in 
complex reactor physics and significant modeling challenges 
\cite{petrovic_benchmark_2021}. 
To further understand and address the technical challenges associated with the 
\gls{AHTR} design, I participate in the \gls{OECD}-\gls{NEA}'s \gls{FHR} 
benchmarking exercise \cite{petrovic_benchmark_2021}.

\section{Objectives and Outline}
This dissertation's objectives are split into three components. 
\begin{enumerate}
    \item Furthering our understanding of the \gls{AHTR} design's complexities 
    through neutronics and thermal-hydraulics modeling 
    \item Creating an open-source tool which enables nuclear reactor design 
    evolutionary algorithm optimization for non-conventional reactor geometries and fuel 
    distributions
    \item Applying the optimization tool to the \gls{AHTR} design 
\end{enumerate}
Chapter \ref{chap:fhr-benchmark} addresses objective 1, chapter \ref{chap:rollo} 
address objective 2, and chapters \ref{chap:method}, \ref{chap:ahtr-plank-opt-results}, 
and \ref{chap:ahtr-assem-opt-results} address objective 3. 

Chapter 2 presents a literature review that organizes and reports on previous 
relevant work. 
I provide an overview of the \gls{FHR} concept, then detail one specific 
\gls{FHR} design: the \gls{AHTR}. 
I describe previous efforts and technical challenges of modeling the \gls{AHTR} design, 
and how these efforts led to the \gls{OECD} \gls{NEA}'s \gls{FHR} benchmark initiation.
Next, I outline additive manufacturing's history and describe the current 
research on using additive manufacturing for nuclear reactor component fabrication. 
I review previous nuclear reactor design optimization efforts and describe how 
additive manufacturing of nuclear reactor components enables optimization for 
less constrained reactor geometries. 
I describe optimization methods that can be leveraged to find optimal reactor 
designs in the expanded design space.
Finally, I give a background of the evolutionary algorithms and detail a specific 
evolutionary algorithm: the genetic algorithm and how it works to conduct global 
optimization robustly.

Chapter 3 describes the \gls{FHR} benchmark specifications and the \gls{UIUC} team's 
results.
The \gls{OECD}-\gls{NEA} and \gls{Georgia Tech} initiated the \gls{FHR} 
benchmark for the \gls{AHTR} design in 2019 \cite{petrovic_benchmark_2021} 
to address \gls{AHTR} modeling challenges such as multiple heterogeneity and 
material cross-section data. 
\gls{UIUC} participates in the \gls{FHR} benchmark with the OpenMC Monte Carlo code 
\cite{romano_openmc_2013} using the ENDF/B-VII.1 material cross section library 
\cite{chadwick_endf/b-vii.1_2011}.
The \gls{UIUC} team consists of myself and my advisors, Professor Kathryn Huff and Dr.
Madicken Munk. 
The results presented are for Phases I-A and I-B which model a steady-state and 
depletion in a 2D \gls{AHTR}, and preliminary multiphysics simulation results. 

Chapter 4 describes the \gls{ROLLO} tool designed for this dissertation. 
\gls{ROLLO} is a Python package that applies evolutionary algorithm 
techniques to optimize nuclear reactor design. 