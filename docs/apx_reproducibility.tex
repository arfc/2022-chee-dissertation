\chapter{Appendix xx: Reproducibility} 

Reproducible science refers to the concept that all research outputs should be 
repeatable using a complete computational environment consisting of the software 
application, input files, and data results \cite{novak_multiscale_2020}.
This section describes the complete computational environment used to obtain the 
results in this dissertation.

The UIUC team's \gls{FHR} benchmark results described in Chapter \ref{chap:fhr-benchmark} 
are available on Github at: 
\begin{equation*}
\mbox{\url{https://github.com/arfc/fhr-benchmark}}
\end{equation*}
Instructions on how to reproduce the benchmark results are included in READMEs within 
the \texttt{fhr-benchmark} directory. 

The \gls{ROLLO} project described in Chapter \ref{chap:rollo} is hosted on Github at: 
\begin{equation*}
    \mbox{\url{https://github.com/arfc/rollo}}
\end{equation*}
Instructions on how to install the \gls{ROLLO} Python package and documentation are 
provided in the README. 

The Latex source files and figures used to compile this dissertation document, and 
the input files and data results for the AHTR temperature model and optimization 
simulations are available on Github at:  
\begin{equation*}
    \mbox{\url{https://github.com/arfc/2022-chee-dissertation}}
\end{equation*}

Table xx provides paths for the tests, input files, and data files referenced in this 
dissertation in order of appearance. 

\begin{landscape}
\begin{table}[htbp!]
    \centering
    \onehalfspacing
    \caption{Paths to tests, input files, and data files for the simulations performed 
    in this dissertation. }
    \label{tab:reproducibility}
    \footnotesize
    \begin{tabular}{p{2cm}p{5cm}p{5cm}}
    \toprule
    \textbf{Section} & \textbf{Description} & \textbf{Location} \\
    \midrule
    \ref{sec:fhr-benchmark-results-ia} & FHR Benchmark Phase I-A Input Files & 
    \texttt{fhr-benchmark/phase1a} \\ 
    \midrule
    \ref{sec:fhr-benchmark-results-ib} & FHR Benchmark Phase I-B Input Files & 
    \texttt{fhr-benchmark/phase1b} \\ 
    \bottomrule
    \end{tabular}
\end{table}
\end{landscape}