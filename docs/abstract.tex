% one paragraph each 

% A brief statement of the motivations and/or issues associated with the research 
% what problem did you study and why is it important? 
Wide-spread adoption of additive manufacturing methods in the nuclear industry 
could drastically reduce reactor fabrication costs and deployment timelines 
and improve reactor safety. 
Additive manufacturing of reactor core components removes the geometric constraints
required by conventional manufacturing, such as slabs as fuel planks and cylinders 
as fuel rods, enabling further optimization and improvement of core geometries. 
Fully benefitting from the new ability to 3D print reactor components requires further 
research into reactor generative design optimization. 
Generative reactor design optimization for arbitrary geometries enabled by additive 
manufacturing is a new concept, and few research demonstrations have been done to 
explore the large new design space. 

% A short description of the methods used. 
% what methods did you use?
In this dissertation, I apply evolutionary algorithm optimization to conduct generative 
\gls{FHR} design.

% A summary of key results obtained 
% what were your principal results? 

% a statement of the implications of the key results (Close the loop)
% what conclusions can you draw from the results,or what are the imploications of 
% your results