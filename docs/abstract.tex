% one paragraph each 

% A brief statement of the motivations and/or issues associated with the research 
% what problem did you study and why is it important? 
Additive manufacturing of reactor core components removes the geometric constraints
required by conventional manufacturing, such as fuel plank slabs and fuel rod 
cylinders, enabling further optimization and improvement of core geometries. 
Wide-spread adoption of additive manufacturing methods in the nuclear industry 
could drastically decrease reactor fabrication costs, reduce deployment timelines, 
and improve reactor safety. 
Due to the expansion of the potential design space facilitated through additive 
manufacturing, reactor designers need to find methods, such as generative 
design, to explore the design space efficiently.
Generative design is an iterative design exploration process; designers input design 
goals and constraints into a generative design software and the software explores 
all the possible permutations of a solution, quickly generating design alternatives. 
Fully benefitting from the new ability to 3D print reactor components requires further 
research into generative reactor design optimization.  
Generative reactor design optimization for arbitrary geometries enabled by additive 
manufacturing is a new concept, and few research demonstrations have been done to 
explore the large new design space. 

% A short description of the methods used. 
% what methods did you use?
In this dissertation, I apply evolutionary algorithms to conduct generative \gls{AHTR} 
design optimization. 
First, I participated in the \gls{OECD} \gls{NEA} \gls{FHR} benchmark to further our 
understanding of the \gls{AHTR} design's complexities.  
Next, I created the \gls{ROLLO} Python package tool that enables reactor design 
evolutionary algorithm optimization for non-conventional reactor geometries and fuel 
distributions. 
I then applied \gls{ROLLO} to conduct generative reactor design optimization
for \gls{AHTR} plank and one-third assembly models.
\gls{ROLLO} generated \gls{AHTR} designs with varying fuel amounts, fuel 
distributions, and coolant channel shapes that optimize for three key reactor 
performance metrics: minimize total fuel amount ($PF_{total}$), maximize heat 
transfer ($T_{max}$), and minimize power peaking ($PPF_{fuel}$).

% A summary of key results obtained 
% what were your principal results? 
This dissertation reported the \gls{FHR} benchmark Phase I-A and I-B results, 
demonstrating the \gls{AHTR}'s passive safety behavior with 
negative temperature coefficients.
A comparison of $k_{eff}$ results between the reference case and the \gls{AHTR} 
configuration with high heavy metal loading demonstrated that increased fuel 
packing does not always correspond with increased $k_{eff}$ due to self-shielding 
effects.
This dissertation also demonstrated that for the \gls{AHTR} full assembly temperature 
model, the temperature distribution peaked in the fuel stripes near the spacers, 
highlighting to reactor designers that spacer material and location in the 
\gls{AHTR} geometry impact temperature peaks.  

This dissertation reported the \gls{AHTR} plank and one-third assembly optimization 
results. 
I characterized each objective's driving factors and relationship with each input 
parameter from the results. 
The final and largest optimization problem is the one-third assembly multi-objective 
optimization that minimized all three objectives ($PF_{total}$, $T_{max}$, and 
$PPF_{fuel}$) while varying all the input parameters ($PF_{total}$, TRISO distribution, 
and coolant channel shape), also known as simulation a-3b-256. 
Simulation a-3b-256 ran for $\sim 3300$ node-hours on the Theta supercomputer with 
8 generations and 256 reactor models per generation, and 
demonstrated 38 one-third assembly reactor models on its Pareto front that met all 
three objectives. 
The reactor models on the Pareto Front have different $PF_{total}$, TRISO distributions, 
and coolant channel shapes, depending on the extent each objective is minimized due 
to the nature of multi-objective optimization that results in a tradeoff between 
objectives. 
The results demonstrated \gls{ROLLO}'s success in conducting multi-objective generative 
reactor design optimization. 

% a statement of the implications of the key results (Close the loop)
% what conclusions can you draw from the results,or what are the imploications of 
% your results
Through participation in the \gls{FHR} benchmark, this dissertation contributes to 
deepening our understanding of the promising \gls{AHTR} technology. 
By designing the \gls{ROLLO} tool and demonstrating \gls{ROLLO}'s success in 
optimization of the \gls{AHTR} beyond classical input parameters, this dissertation 
contributes to optimization tool development for reactors of the future. 
As additive manufacturing technology advances and the \gls{TCR} program 
demonstrates the first 3D printed operational reactor, more reactor designers 
will begin to explore the vast design space enabled by 3D printing. 
\gls{ROLLO} can be utilized to optimize other reactor types for arbitrary
geometries and parameters, enabling further optimization and improvement of reactor 
geometries.