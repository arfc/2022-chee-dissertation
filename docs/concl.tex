\chapter{Conclusions and Future Work}
\glsresetall
\label{chap:concl}

Additive manufacturing of reactor core components removes the geometric constraints
of conventional manufacturing, which will reduce reactor fabrication costs, 
deployment timelines, and improve reactor safety. 
Fully benefitting from the new ability to 3D print reactor components, requires further 
research into reactor generative design optimization. 
This dissertation explores this new design space by designing and applying the flexible 
and open-source \gls{ROLLO} tool to optimize the \gls{AHTR} for non-conventional 
geometries and parameters. 
I successfully explored the \gls{AHTR}'s arbitrary geometry design space by completing 
these three dissertation objectives: 
\begin{enumerate}
    \item I furthered our understanding of the \gls{AHTR} design's complexities 
    through neutronics and thermal-hydraulics modeling 
    \item I created the open-source \gls{ROLLO} tool that enables nuclear reactor design 
    evolutionary algorithm optimization for non-conventional reactor geometries and fuel 
    distributions
    \item I applied \gls{ROLLO} to the \gls{AHTR} design 
\end{enumerate}

% structure 
% An introductory restatement of research problem, aims and/or research question
% A summary of findings and limitations
% Practical applications/implications
% Recommendations for further research

% talk about ROLLO method 
% talk about AHTR geometry recommendations 
% acknowledge limitations
% this type of coolant variation is not as significant as TRISO distribution variation. 
% first steps towards true generative design. 

\section{Future Work}
% characterize hyperparameter space better for FHR assem multi obj 

% further exploration of the AHTR arbitrary geometry design space. Represent as pixels? 
% or some sort of smarter expression... see structural aerospace design space. 
% especially once we have more compute time. 