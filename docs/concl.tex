\chapter{Conclusions and Future Work}
\glsresetall
\label{chap:concl}

Additive manufacturing of reactor core components removes the geometric constraints
required by conventional manufacturing, such as slabs as fuel planks and cylinders 
as fuel rods, enabling further optimization and improvement of core geometries. 
Wide-spread adoption of additive manufacturing methods in the nuclear industry
could drastically reduce fabrication costs and deployment timelines, and improve 
reactor safety. 
Fully benefitting from the new ability to 3D print reactor components requires further 
research into reactor generative design optimization. 
This dissertation explores the new design space enabled by additive manufacturing 
by designing and applying the flexible and open-source \gls{ROLLO} tool to optimize 
the \gls{AHTR} for non-conventional geometries and parameters. 
I successfully explored the \gls{AHTR}'s arbitrary geometry design space by completing 
these three dissertation objectives: 
\begin{enumerate}
    \item I furthered our understanding of the \gls{AHTR} design's complexities 
    through neutronics and thermal-hydraulics modeling by participating in the 
    \gls{OECD} \gls{NEA} \gls{FHR} benchmark.
    \item I developed the open-source \gls{ROLLO} tool that enables generative reactor 
    design using evolutionary algorithm optimization for non-conventional reactor 
    geometries and fuel distributions.
    \item I applied \gls{ROLLO} to conduct generative \gls{AHTR} design.
    \gls{ROLLO} generated \gls{AHTR} designs with varying fuel amounts, fuel 
    distributions, and coolant channel shapes that optimize for three key reactor 
    performance metrics: minimize total fuel amount, maximize heat transfer, and 
    minimize power peaking.
\end{enumerate}
Chapter \ref{chap:fhr-benchmark} addressed objective 1, chapter \ref{chap:rollo} 
addressed objective 2, and chapters \ref{chap:method}, \ref{chap:ahtr-plank-opt-results}, 
and \ref{chap:ahtr-assem-opt-results} addressed objective 3. 

Chapter \ref{chap:fhr-benchmark} reported the \gls{FHR} benchmark Phase I-A and I-B 
results, highlighting the \gls{AHTR}'s passive safety behavior with 
negative temperature coefficients. 
Comparison of $k_{eff}$ results between the reference case and the \gls{AHTR} 
configuration with high heavy metal loading demonstrated that increased fuel 
packing does not always correspond with increased $k_{eff}$ due to self-shielding 
effects.
Chapter \ref{chap:fhr-benchmark} also reported the results of the \gls{AHTR} full 
assembly multiphysics model. The temperature distribution peaked in the fuel stripes near 
the spacers, highlighting to reactor designers that spacer material and location in the 
\gls{AHTR} geometry impact temperature peaks.  

Chapter \ref{chap:rollo} described the \gls{ROLLO} tool developed for this 
dissertation. 
\gls{ROLLO} is a Python package that applies evolutionary algorithm 
optimization techniques to generate nuclear reactor designs that meet user-defined 
objectives and constraints based on user-defined input value ranges. 
\gls{ROLLO} enables reactor designers to optimize any reactor model using robust 
evolutionary algorithm methods without going through the cumbersome process of setting up 
an evolutionary algorithm framework, selecting appropriate hyperparameters, and 
setting up parallelization.
\gls{ROLLO} is effective, flexible, open-source, parallel, reproducible, usable, and 
hosted on Github \cite{chee_rollo_2021}. 

% add more description to contextualize more so that the conclusion can stand on its own
Chapter \ref{chap:method} described the modeling and optimization methodology of the 
\gls{AHTR} plank and one-third assembly optimization conducted using the \gls{ROLLO} 
software.
I varied the following \gls{AHTR} plank and one-third assembly input parameters: 
\gls{TRISO} packing fraction distribution ($\rho_{TRISO}(\vec{r})$), total fuel 
packing fraction ($PF_{total}$), and coolant channel shape; in an effort to minimize 
the following objectives: total fuel packing fraction ($PF_{total}$), maximum 
temperature ($T_{max}$), and fuel-normalized power peaking factor ($PPF_{fuel}$). 

Chapter \ref{chap:ahtr-plank-opt-results} reported the \gls{AHTR} plank's 
\gls{ROLLO} optimization results.
I characterized each objective's driving factors and relationship 
with each input parameter from the results. 
I determined that the minimize $PF_{total}$ objective is driven by maximizing the plank's 
total fission reaction rate and influences oscillations in the TRISO distribution to 
achieve the objective. 
I determined that the minimize $PPF_{fuel}$ objective is driven by flattening the plank's
thermal flux distribution and influences $PF_{total}$ and oscillations in the TRISO 
distribution to achieve the objective.
I determined that the minimize $T_{max}$ objective flattens TRISO distribution and 
maximizes coolant channel shape's radius values to achieve the objective.
Characterizations of each objective for the simple \gls{AHTR} plank model provided 
insights for Chapter \ref{chap:ahtr-assem-opt-results}'s multi-objective 
optimization for the more complex \gls{AHTR} one-third assembly model. 

Chapter \ref{chap:ahtr-assem-opt-results} reported the \gls{AHTR} one-third assembly's
\gls{ROLLO} optimization results.
I verified that the one-third assembly objectives follows the same driving 
factors as the \gls{AHTR} plank optimization objectives. 
The results demonstrated that the minimize $PF_{total}$ objective's driving factor 
maximize total fission reaction rate and minimize $PPF_{fuel}$ objective's driving 
factor flattening thermal flux distribution influenced each other resulting in unexpected 
TRISO distributions at different $PF_{total}$ values. 
Further optimization of the \gls{AHTR} design will benefit from awareness 
of the objectives' relationship. 
The results also demonstrated that the coolant channel shape variation did 
not have as high of an impact on $T_{max}$ as \gls{TRISO} distribution variation.
Simulation a-3b-256's multi-objective optimization showed the result of minimizing all 
three objectives ($PF_{total}$, $T_{max}$, and $PPF_{fuel}$) while varying 
all the input parameters ($PF_{total}$, TRISO distribution, and coolant channel shape).
Figure \ref{fig:assem-obj-3-all-256} showed the 38 reactor models on simulation 
a-3b-256's Pareto front that met all three objectives. 
The reactor models on the Pareto Front have different $PF_{total}$, TRISO distributions, 
and coolant channel shapes, depending on the extent each objective is minimized due 
to the nature of multi-objective optimization that results in a tradeoff between 
objectives. 
These results demonstrate that \gls{ROLLO}'s generative design optimization process 
provides the reactor designer with a set of equally good reactor models.
From this point, it is up to the reactor designer to determine the importance of each 
objective for their purposes, then conduct further sensitivity analysis and 
use higher fidelity models to study the optimal design space before selecting 
the final reactor model.

Chapters \ref{chap:ahtr-plank-opt-results} and \ref{chap:ahtr-assem-opt-results} 
demonstrated \gls{ROLLO}'s success in conducting multi-objective generative reactor 
design optimization. 
\gls{ROLLO} conducted a global search of the large reactor design space and successfully 
generated optimal reactor models on the Pareto front that satisfy all the objectives. 
\gls{ROLLO} also showed how sensitive each input parameter is in relation to 
the objectives. 
Once the \gls{ROLLO} search is complete, reactor designers gain a better intuition of 
the model's reactor physics and can view the narrower reactor design space that meets 
their defined objectives.   

Through participation in the \gls{FHR} benchmark, this dissertation contributes to 
deepening our understanding of the promising \gls{AHTR} technology. 
By designing the \gls{ROLLO} tool and demonstrating \gls{ROLLO}'s success in 
optimization of the \gls{AHTR} beyond classical input parameters, this dissertation 
contributes to optimization tool development for reactors of the future with
arbitrary geometries and parameters. 


\section{Future Work}
This dissertation's generative design optimization of the \gls{AHTR} model used sine 
distribution variations to govern the \gls{TRISO} packing fraction distribution and  
varied cylinder radius' to generate varying sinusoidal-like pattern coolant channel 
shapes. 
These input parameter variations are only one way to represent the \gls{AHTR} 
geometry. 
There are many other ways to represent the geometry that might give \gls{ROLLO} more 
freedom to explore a larger design space and generate more optimized designs. 
Future reactor designers could consider topology optimization to optimize coolant 
channel shape further.
The major limitation is the computational cost of modeling fluid flow in complex 
coolant channel designs. 

This dissertation demonstrated \gls{ROLLO}'s success in conducting multi-objective 
generative reactor design optimization. 
\gls{ROLLO} can be easily used to optimize any reactor type for any perceivable 
arbitrary input parameters. 
As additive manufacturing technology advances and the \gls{TCR} program 
demonstrates the first 3D printed operational reactor, more reactor designers 
will begin to explore the vast design space enabled by 3D printing. 
Future work includes utilizing \gls{ROLLO} to optimize other reactor types. 

