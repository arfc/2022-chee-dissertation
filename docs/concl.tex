\chapter{Conclusions and Future Work}
\glsresetall
\label{chap:concl}

Additive manufacturing of reactor core components removes the geometric constraints
required by conventional manufacturing. 
This reduces reactor fabrication costs, deployment timelines, and improves reactor 
safety. 
Fully benefitting from the new ability to 3D print reactor components, requires further 
research into reactor generative design optimization. 
This dissertation explores this new design space by designing and applying the flexible 
and open-source \gls{ROLLO} tool to optimize the \gls{AHTR} for non-conventional 
geometries and parameters. 
I successfully explored the \gls{AHTR}'s arbitrary geometry design space by completing 
these three dissertation objectives: 
\begin{enumerate}
    \item I furthered our understanding of the \gls{AHTR} design's complexities 
    through neutronics and thermal-hydraulics modeling by participating in the 
    \gls{OECD} \gls{NEA} \gls{FHR} benchmark.
    \item I developed the open-source \gls{ROLLO} tool that enables generative reactor 
    design evolutionary algorithm optimization for non-conventional reactor geometries 
    and fuel distributions.
    \item I applied \gls{ROLLO} to optimize the \gls{AHTR} plank and one-third assembly
    designs by varying fuel amounts, fuel distributions, and coolant channel shapes to 
    minimize three key reactor performance metrics: total fuel amount, maximum 
    temperature, and fuel-normalized power peaking factor.
\end{enumerate}
Chapter \ref{chap:fhr-benchmark} addressed objective 1, chapter \ref{chap:rollo} 
addressed objective 2, and chapters \ref{chap:method}, \ref{chap:ahtr-plank-opt-results}, 
and \ref{chap:ahtr-assem-opt-results} addressed objective 3. 

Chapter \ref{chap:fhr-benchmark} reported the \gls{FHR} benchmark Phase I-A and I-B 
results, which highlighted the \gls{AHTR}'s passive safety behavior with 
negative temperature coefficients. 
Comparison of $k_{eff}$ results between the reference case and the \gls{AHTR} 
configuration with high heavy metal loading demonstrated that increased fuel 
packing does not always correspond with increased $k_{eff}$ due to self-shielding 
effects.
Chapter \ref{chap:fhr-benchmark} also reported the \gls{AHTR} full assembly 
multiphysics model results. The temperature distribution peaks in the fuel stripes near 
the spacers, highlighting to reactor designers that spacer material and location in the 
\gls{AHTR} geometry impact temperature peaks.  

Chapter \ref{chap:rollo} described the \gls{ROLLO} optimization tool developed for this 
dissertation. 
\gls{ROLLO} is a Python package that applies evolutionary algorithm 
optimization techniques to generate nuclear reactor designs that meet user-defined 
objectives and constraints based on user-defined input value ranges. 
\gls{ROLLO} enables reactor designers to optimize any reactor model using robust 
evolutionary algorithm methods without going through the cumbersome process of setting up 
a genetic algorithm framework, selecting appropriate hyperparameters, and setting up
parallelization.
\gls{ROLLO} is effective, flexible, open-source, parallel, reproducible, usable, and 
hosted on Github \cite{chee_rollo_2021}. 

Chapter \ref{chap:method} described the modeling and optimization methodology of the 
\gls{AHTR} plank and one-third assembly optimization for non-conventional 
geometries and parameters conducted using the \gls{ROLLO} software.
I varied the following \gls{AHTR} plank and one-third assembly input parameters: 
\gls{TRISO} packing fraction distribution ($\rho_{TRISO}(\vec{r})$), total fuel 
packing fraction ($PF_{total}$), and coolant channel shape; in an effort to minimize 
the following objectives: total fuel packing fraction ($PF_{total}$), maximum 
temperature ($T_{max}$), and fuel-normalized power peaking factor ($PPF_{fuel}$). 

Chapter \ref{chap:ahtr-plank-opt-results} reported the \gls{AHTR} plank's 
\gls{ROLLO} optimization results.
I characterized each objective's driving factors and relationship with each input 
parameter. 
I determined that the minimize $PF_{total}$ objective is driven by maximizing the plank's 
total fission reaction rate and influences oscillations in the TRISO distribution to 
achieve the objective. 
I determined that the minimize $PPF_{fuel}$ objective is driven by flattening the plank's
thermal flux distribution and influences $PF_{total}$ and oscillations in the TRISO 
distribution to achieve the objective.
I determined that the minimize $T_{max}$ objective flattens TRISO distribution and 
maximizes coolant channel shape's radius values to achieve the objective.
Characterizations of each objective for a simple \gls{AHTR} plank model provided 
insights for Chapter \ref{chap:ahtr-assem-opt-results}'s multi-objective 
optimization for the \gls{AHTR} one-third assembly model. 

Chapter \ref{chap:ahtr-assem-opt-results} reported the \gls{AHTR} one-third assembly's
\gls{ROLLO} optimization results.
I verified that each of the one-third assembly objective follows the same driving 
factors as the \gls{AHTR} plank optimization objectives. 


% this type of coolant channel shape variation might not work well. 

Chapters \ref{chap:ahtr-plank-opt-results} and \ref{chap:ahtr-assem-opt-results} 
demonstrated \gls{ROLLO}'s success in conducting multi-objective generative reactor 
design optimization. 
\gls{ROLLO} conducted a global search of the large reactor design space, and successfully 
generated optimal reactor models on the Pareto front that satisfy all the objectives. 
Once the \gls{ROLLO} search is complete, reactor designers gain a better intuition of 
the model's reactor physics and can view the narrower reactor design space that meets 
their defined objectives. 
\gls{ROLLO} can be applied to any perceivable reactor model.  

% structure 
% An introductory restatement of research problem, aims and/or research question
% A summary of findings and limitations
% Practical applications/implications
% Recommendations for further research

% talk about ROLLO method 
% talk about AHTR geometry recommendations 
% acknowledge limitations
% this type of coolant variation is not as significant as TRISO distribution variation. 
% first steps towards true generative design. 

\section{Future Work}
% characterize hyperparameter space better for FHR assem multi obj 

% further exploration of the AHTR arbitrary geometry design space. Represent as pixels? 
% or some sort of smarter expression... see structural aerospace design space. 
% especially once we have more compute time. 

% first talk about how this model can be further developed 
% next talk about the potenial for the bigger picture of integrated generative design tools 