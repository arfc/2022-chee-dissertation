\section{PPF and Flux Study}
Equation \ref{eq:reaction-rate-fission} shows the relationship between fission reaction rate, 
flux, and material properties. 
\begin{align}
\label{eq:reaction-rate-fission}
    RR_f &= \Phi \times \sigma_f \times N \\
\intertext{where}
    RR_f &= \mbox{fission reaction rate } [reactions \cdot cm^{-3} \cdot s^{-1}]\nonumber \\
    \Phi &= \mbox{neutron flux } [neutrons \cdot cm^{-2} \cdot s^{-1}] \nonumber \\
    \sigma_f &= \mbox{microscopic cross section } [cm^2] \nonumber \\
    N &= \mbox{atomic number density } [atoms \cdot cm^{-3}] \nonumber 
\end{align}

Since microscopic cross section is constant for the same fuel material, we can rearrange 
Equation \ref{eq:reaction-rate-fission} into Equation \ref{eq:reaction-rate-fission-prop}. 
The fission reaction rate ($RR_f$) is proportional to fission energy production rate \\
(\texttt{fission-q-recoverable} $[eV/src]$) used to calculate fuel-normalized power peaking 
factor ($PPF_{fuel}$). 
Section \ref{sec:ahtr_slab_output} details how I calculated $PPF_{fuel}$. 
The atomic number density (N) is proportional to the fuel packing fraction. 
We can further rearrange Equation \ref{eq:reaction-rate-fission-prop} into 
Equation \ref{eq:flux-prop-fqr}
\begin{align}
    \label{eq:reaction-rate-fission-prop}
    \Phi \propto \frac{RR_f}{N}
\end{align}
\begin{align}
    \label{eq:flux-prop-fqr}
    \Phi \propto \frac{fqr}{PF}
\end{align}

Therefore, a flatter plank's thermal flux will result in a lower fuel-normalized power peaking 
factor. 