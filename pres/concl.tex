\subsection{Conclusion}
\begin{frame}
    \frametitle{Conclusion}
    \visible<1->{\begin{block}{AHTR Model Development for the FHR Benchmark}
        \small
        \textbf{Relevance}
        \begin{itemize}
            \item The triple heterogeneity introduced by the geometrically complex 
            fuel assembly design makes accurate reactor physics simulations challenging
        \end{itemize}}
        \visible<2->{\textbf{Results Presented} 
        \begin{itemize}
            \item FHR benchmark Phase I-A and I-B results
            \item AHTR full assembly temperature model 
        \end{itemize}}
        \visible<3->{\textbf{Major Takeaways} 
        \begin{itemize}
            \item AHTR has passive safety behavior with negative temperature coefficients
            \item Increased fuel packing does not always correspond with increased 
            $k_{eff}$ due to self-shielding effects 
            \item AHTR temperature peaks in the fuel stripes near the spacers 
            \item FHR benchmark results gives confidence to the AHTR base model's accuracy
        \end{itemize}}
    \end{block}

    \visible<4->{Through participation in the FHR benchmark, this dissertation 
    contributes to \textbf{deepening our understanding of the promising \gls{AHTR} 
    technology}.} 
\end{frame}

\begin{frame}
    \frametitle{contd. Conclusion}
    \visible<1->{\begin{block}{ROLLO Tool Development and AHTR Non-Conventional Design Optimization}
        \small
        \textbf{Relevance}
        \begin{itemize}
            \item Optimization tools for generating new reactor designs enabled by
            3D printing do not exist
            \item Few demonstrations of reactor optimization for non-conventional 
            geometries and parameters exist
        \end{itemize}}
        \visible<2->{\textbf{Results Presented}
        \begin{itemize}
            \item \acrfull{ROLLO} tool 
            \item AHTR Optimization for heterogeneous fuel distributions and wavy 
            coolant channels 
        \end{itemize}}
        \visible<3->{\textbf{Major Takeaways} 
        \begin{itemize}
            \item ROLLO successfully conducted multi-objective generative reactor 
            design optimization generating optimal reactor models on a Pareto front 
            that satisfy all the user-defined objectives
        \end{itemize}}
    \end{block}
    \vspace{-0.1cm}
    \visible<4->{By designing the ROLLO tool and demonstrating its success in 
    optimization of the \gls{AHTR} beyond classical input parameters, this dissertation 
    contributes to \textbf{optimization tool development for reactors of the future}.} 
\end{frame}

\subsection{Future Work}
\begin{frame}
    \frametitle{Future Work}
    \begin{block}{Potential Future Research Efforts}
    \begin{enumerate}
        \item Completion of FHR benchmark Phases I-C, II, III
        \begin{itemize}
            \item Reveals new interesting AHTR physics phenomena 
        \end{itemize}
        \item ROLLO software improvements 
            \begin{itemize}
                \item Continuous improvement will enable more users to benefit from ROLLO
            \end{itemize}
        \item AHTR design optimization with other geometry representations 
            \begin{itemize}
                \item Topology optimization for coolant channel shape 
                \item Enables further exploration of the AHTR design space 
            \end{itemize}
        \item Generative reactor design that takes into account more multiphysics phenomena 
            \begin{itemize}
                \item AHTR model in this work: beginning of life neutronics and temperature 
                \item Other multiphysics components: burnup effects, structural mechanics 
                thermal stresses, chemistry, high-fidelity thermal-hydraulics 
                \item Enables further optimized reactor designs that take into account more 
                multiphysics components 
            \end{itemize}
        \item Generative reactor design for other reactor types
    \end{enumerate}
    \end{block}
\end{frame}