\subsection*{Grad School Journey}
\begin{frame}
    \frametitle{Grad School Journey}
    \vspace{-0.2cm}
    \begin{figure}[]
        \includegraphics[width=0.9\linewidth]{figures/grad-school-journey.png} 
    \end{figure}
\end{frame}

\subsection*{Summary}
\begin{frame}
    \frametitle{Summary}
    \begin{columns}[t]
        \begin{column}{0.5\linewidth}
            \begin{block}{AHTR Model Development for the FHR Benchmark}
                \textbf{Results Presented} 
                \begin{itemize}
                    \item FHR benchmark Phase I-A and I-B results
                    \item AHTR full assembly temperature model 
                \end{itemize}

                Through participation in the FHR benchmark, this dissertation contributes to 
                \textbf{deepening our understanding of the promising \gls{AHTR} technology}.
            \end{block}
        \end{column}
        \begin{column}{0.5\linewidth}
            \begin{block}{ROLLO Tool Development and AHTR Non-Conventional Design Optimization}
                \textbf{Results Presented}
                \begin{itemize}
                    \item \acrfull{ROLLO} tool 
                    \item AHTR Optimization for heterogeneous fuel distributions and wavy 
                    coolant channels 
                \end{itemize}
                
                By designing the ROLLO tool and demonstrating its success in 
                optimization of the \gls{AHTR} beyond classical input parameters, this dissertation 
                contributes to \textbf{optimization tool development for reactors of the future}.
            \end{block}
        \end{column}
    \end{columns}
\end{frame}

\subsection*{FHR Benchmark Specifications}
\begin{frame}
    \frametitle{FHR Benchmark Specifications}
    UIUC participates in the benchmark with OpenMC and using the ENDF/B-VII.1 material 
    cross section library
    \vspace{-0.2cm}
    \begin{table}
        \caption{OECD NEA's FHR Benchmark Phases 
        \cite{petrovic_benchmark_2021}.}
        \vspace{-0.25cm}
        \includegraphics[width=0.7\linewidth]{figures/benchmark-phases.png} 
    \end{table}
    \vspace{-0.3cm}
    \begin{figure}[]
        \includegraphics[width=0.27\linewidth]{../docs/figures/ahtr-fuel-element.png} 
        \vspace{-0.2cm}
        \caption{AHTR fuel assembly.}
    \end{figure}
\end{frame}

\begin{frame}
    \frametitle{FHR Benchmark Specifications}
    Only Phase I-A and I-B specifications have been released 
    \begin{table}
        \caption{Description of the \acrlong{FHR} benchmark Phase I-A cases 
        \vspace{-0.25cm}
        \cite{petrovic_benchmark_2021}.}
        \includegraphics[width=0.6\linewidth]{figures/benchmark-cases.png} 
    \end{table}
    Benchmark participants must produce the following results for 
    the 9 cases: $k_{eff}$, reactivity coefficients ($\beta_{eff}$, 
    $\alpha_D$, $\alpha_{T, FliBe}$, $\alpha_M$), fission source distribution, 
    neutron flux distribution, fuel assembly averaged neutron spectrum
\end{frame}

\begin{frame}
    \frametitle{FHR Benchmark Phase I-B Results}
    \begin{itemize}
        \item FHR Benchmark Phase I-B: 2D assembly depletion model
        \item Benchmark participants are working on resolving differences in 
        these results
    \end{itemize}
    \vspace{-0.2cm}
    \begin{figure}[]
        \centering
        \includegraphics[width=0.75\linewidth]{../docs/figures/phase1b_keff.png} 
        \caption{UIUC results: FHR Benchmark Phase I-B depletion 
        $k_{eff}$ evolution for Cases 1B, 4B, and 7B. Case 1B is the reference case, 
        Case 4B is the discrete \acrlong{BP} case, and Case 7B is the 19.75$\%$ 
        enrichment case. Error bars are included but are barely visible due to the 
        low $\sim$40pcm uncertainty.}
    \end{figure}
\end{frame}

\subsection*{Key Neutronics Parameters Verification}
\begin{frame}
    \frametitle{AHTR Temp Model $k_{eff}$ and Reactivity Coefficients Verification}
        \begin{table}
            \caption{$k_{eff}$ and reactivity comparison.}
            \vspace{-0.2cm}
            \includegraphics[width=0.9\linewidth]{figures/benchmark-keff.png}
        \end{table}
        The 13pcm $k_{eff}$ and 6pcm reactivity diff, between
        continuous and homogenized OpenMC simulations are within uncertainty, showing 
        that \textbf{selected spatial homogenizations and energy discretizations are 
        acceptable.}
        The Moltres simulation shows a 423pcm diff in $k_{eff}$ and 216pcm 
        diff in reactivity.
        \begin{table}
            \caption{Reactivity coefficients comparison.}
            \includegraphics[width=0.85\linewidth]{figures/benchmark-coeff.png}
        \end{table}
        \textbf{Good agreement} for Moltres' delayed neutron fraction ($\beta_{eff}$) and 
        temperature reactivity feedback ($\frac{\Delta \rho}{\Delta T}$)
\end{frame}

\begin{frame}
    \frametitle{AHTR Temp Model Flux Verification}
    \begin{columns}
    \begin{column}{0.7\textwidth}
    \begin{figure}[]
        \centering
        \includegraphics[width=\linewidth]{figures/benchmark-flux.png} 
        \caption{4-group flux distribution comparison.}
    \end{figure}
    \end{column}
    \begin{column}{0.3\textwidth}
        2-norm Diff [\%]
        \begin{itemize}
            \item Group 1: 0.13\% 
            \item Group 2: 0.08\% 
            \item Group 3: 0.10\% 
            \item Group 4: 0.09\%
        \end{itemize}
        Max Diff [\%]
        \begin{itemize}
            \item Group 1: -10.57\% 
            \item Group 2: +7.58\% 
            \item Group 3: +8.96\% 
            \item Group 4: +6.97\%
        \end{itemize}
    \end{column}
    \end{columns}
\end{frame}

\begin{frame}
    \frametitle{AHTR Temp Model Neutron Energy Spectrum Verification}
            \begin{figure}[]
                \centering
                \includegraphics[width=0.75\linewidth]{figures/benchmark-spectrum.png} 
                \caption{Neutron Energy Spectrum Comparison.}
            \end{figure}
        \textbf{Good agreement} between OpenMC and Moltres models 4-group spectrums.
\end{frame}

\subsection{TRISO Particle Homogenization}
\begin{frame}
    \frametitle{TRISO Particle Homogenization}
    \begin{table}
        \caption{Straightened \acrfull{AHTR} fuel plank $k_{eff}$ for the case with 
        no \gls{TRISO} homogenization and case with homogenization of the four outer 
        layers. Both simulations were run on one BlueWaters supercomputer XE Node 
        using OpenMC with 80 active 
        cycles, 20 inactive cycles, and 8000 particles.}
        \includegraphics[width=0.9\linewidth]{figures/triso-homogenization.png} 
    \end{table}

    The \gls{TRISO} particle outer four-layer homogenization resulted in a $30\%$ 
    speed-up without compromising accuracy with $k_{eff}$ values within each 
    other's uncertainty.

    \vspace{0.3cm}
    As a result, the homogenized models are used for all subsequent optimization efforts. 

\end{frame}

\subsection{ROLLO Verification}
\begin{frame}
    \frametitle{ROLLO Successfully Verified with $^{239}Pu$ Critical Bare Sphere}
    \begin{figure}
        \includegraphics[width=0.85\linewidth]{../docs/figures/radius-convergence.png} 
        \caption{Results for each generation for \gls{ROLLO}'s genetic algorithm 
        optimization to the find the critical radius of a  $^{239}Pu$ bare sphere.}
    \end{figure}
    \vspace{-0.2cm}
    \textbf{ROLLO successfully finds the critical radius of the $^{239}Pu$ bare sphere 
    to be 4.9856cm.}
\end{frame}

\begin{frame}
    \frametitle{ROLLO $^{239}Pu$ Critical Bare Sphere Input File}
    \begin{figure}
        \includegraphics[width=0.49\linewidth]{figures/rollo-verify-file.png} 
        \includegraphics[width=0.49\linewidth]{figures/rollo-verify-file2.png}
        \caption{ROLLO $^{239}Pu$ Critical Bare Sphere Input File.}
    \end{figure}
\end{frame}

\begin{frame}
    \frametitle{AHTR Plank Geometry}
    A sine distribution governs TRISO packing fraction distribution: 
    \begin{align}
        \rho_{TRISO}(\vec{x}) &= \left(\textbf{a}\cdot sin(\textbf{b}\cdot x + \textbf{c}) + 2\right) \cdot NF \nonumber
    \end{align}
    \begin{figure}
        \includegraphics[width=0.9\linewidth]{../docs/figures/straightened_plank.png} 
        \caption{Straightened AHTR Plank with 10 fuel cells with random TRISO packing.}
    \end{figure}
    $r_{top}$ and $r_{bot}$ control coolant channel shape: 
    \begin{figure}
        \includegraphics[width=\linewidth]{../docs/figures/coolant-channel-shape.png} 
        \caption{AHTR Plank with coolant channel shape variation, $r_{top}$ = 0.2cm and 
        $r_{bot}$ = 0.3cm.}
    \end{figure}
\end{frame}

\begin{frame}
    \frametitle{Simulation a-3b Hypervolume}
    \begin{table}
        \caption{Simulation a-3b hypervolume values at each generation.}
        \includegraphics[width=0.35\linewidth]{figures/a-3b-hypervolume.png} 
    \end{table}

    For each optimization, I must \textbf{balance convergence and computational cost}.

    \vspace{0.1cm}
    The hypervolume is calculated by finding the volume between the reference point and 
    the objective values of the Pareto front's reactor models (bigger hypervolume = 
    more converged solution).

    \vspace{0.1cm}
    \textbf{I determine if convergence criteria is met by evaluating if the difference between 
    generations' hypervolume values are getting smaller.}
\end{frame}