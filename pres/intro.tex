\subsection{Overview}
    \begin{frame}
        \frametitle{Overview}
        In this defense, I will show that I successfully: 
        \begin{enumerate}
            \item Furthered our understanding of the \gls{AHTR} design's complexities 
            through neutronics and temperature modeling
            \item Created an open-source tool that enables reactor generative 
            design optimization with evolutionary algorithms
            \item Applied the optimization tool to the \gls{AHTR} design for 
            non-conventional geometries and fuel distributions 
        \end{enumerate}
    \end{frame}

\subsection{Background: Advanced High Temperature Reactor}
    \begin{frame}
        \frametitle{MSR + VHTR = FHR}
        Gen IV Forum identified \textbf{new and innovative Gen IV nuclear energy systems}: 
        Molten Salt Reactors (MSR) and Very High-Temperature Reactors (VHTR). 
        \vspace{0.2cm}
        The Fluoride-Salt Cooled High-Temperature Reactor (FHR) concept combines 
        the \textbf{best aspects of MSR and VHTR}.
        \vspace{-0.2cm}
        \begin{columns}
        \begin{column}{0.7\textwidth}
        \begin{figure}[]
            \centering
            \includegraphics[width=\linewidth]{figures/fhr_venn.png} 
            \caption{Venn diagram.}
        \end{figure}
        \end{column}
        \begin{column}{0.3\textwidth}
            TRISO fuel = safety
            \\
            salt-cooled = superior cooling and low operating pressure 
        \end{column}
    \end{columns}
        \end{frame}

    \begin{frame}
    \frametitle{Advanced High Temperature Reactor Design}
        \begin{itemize}
        \item Design developed by Oak Ridge National Laboratory
        \item Prismatic FHR design with 252 hexagonal fuel assemblies consisting of 
        18 fuel planks arranged in 3 diamond-shaped sectors. 
        \end{itemize}
    \begin{figure}[]
        \centering
        \includegraphics[width=0.9\linewidth]{../docs/figures/ahtr.png} 
        \caption{\acrlong{AHTR} fuel assembly (left) and core configuration (right) 
        reproduced from \cite{ramey_monte_2018}.}
        \label{fig:ahtr}
    \end{figure}
    \end{frame}

    \begin{frame}
    \frametitle{Advanced High Temperature Reactor Geometry}
    \begin{figure}[]
        \includegraphics[width=0.45\linewidth]{../docs/figures/ahtr-fuel-element.png} 
        \includegraphics[width=0.45\linewidth]{figures/ahtr-plank.png}
        \caption{AHTR fuel assembly with 18 fuel plates arranged in 
        three diamond-shaped sectors, with a central Y-shaped and external channel 
        graphite structure.}
    \end{figure}
    \vspace{-0.2cm}
    The AHTR fuel has a \textbf{triple heterogeneity}: hexagonal fuel elements with 
    fuel planks, and TRISO particles embedded in stripes within each plank.
    \end{frame}

    \begin{frame}
    \frametitle{FHR Benchmark}
    \begin{itemize}
        \item The AHTR's fuel geometry's triple heterogeneity results in
        \textbf{complex reactor physics} and \textbf{significant modeling challenges}
        \begin{itemize}
            \item many surfaces to model = computationally expensive 
            \item problem homogenization might result in loss of reactor physics effects 
        \end{itemize}
        \item In 2019 the \textbf{OECD-NEA initiated the FHR benchmark}. Its objective 
        is to identify the applicability, accuracy, and practicality of the latest 
        methods and codes to assess the current state of the art for FHR modeling
    \end{itemize}
    \begin{figure}[]
        \includegraphics[width=0.55\linewidth]{figures/benchmark.png} 
        \caption{OECD NEA's FHR Benchmark \cite{petrovic_benchmark_2021}.}
    \end{figure}
    \end{frame}

\subsection{Objectives: AHTR Model Development}
    \begin{frame}
        \frametitle{Research Objectives: AHTR Model Development}
        \begin{block}{Technical Gap}
            The geometrically complex AHTR design is difficult and computationally 
            expensive to model. 
        \end{block}
        \begin{block}{Research Objectives: AHTR Model Development}
        \begin{itemize}
            \item Participate in FHR Benchmark's neutronics modeling to further our 
            understanding of the AHTR design's complexities
            \item AHTR temperature model to capture thermal feedback effects
        \end{itemize}
        \end{block}
        \begin{block}{Link to Reactor Optimization for Non-conventional Designs}
        \begin{itemize}
        \item By participating in the benchmark, I ensure an accurate AHTR base model
        \item Thus, I can expect accurate answers for the optimized AHTR designs
        \end{itemize}
        \end{block}
    \end{frame}

\subsection{Background: Generative Reactor Design Optimization}
\begin{frame}
    \frametitle{Overview}
    In this defense, I will show that I successfully: 
    \begin{enumerate}
        \item Furthered our understanding of the \gls{AHTR} design's complexities 
        through neutronics and temperature modeling
        \item Created an open-source tool that enables reactor generative 
        design optimization with evolutionary algorithms
        \item Applied the optimization tool to the \gls{AHTR} design for 
        non-conventional geometries and fuel distributions 
    \end{enumerate}
\end{frame}

\begin{frame}
    \frametitle{3D Printing a Nuclear Reactor}
    Additive manufacturing could enable us to \textbf{surpass classical 
    manufacturing constraints} and optimize for \textbf{arbitrary geometries and 
    parameters}. 
    \begin{figure}[]
        \includegraphics[width=0.2\linewidth]{figures/wavy-channels.png}
        \caption{Example of a future reactor design with 3D printed wavy flow channels}
    \end{figure}
    Wide-spread adoption of 3D printing for nuclear reactor parts 
    \begin{itemize}
        \item reduce reactor fabrication costs 
        \item reduce deployment timelines 
        \item improve reactor safety 
    \end{itemize}
    \vspace{0.2cm}
    We require methods, such as \textbf{generative design}, to explore the 
    new design space more efficiently. 
\end{frame}

  \begin{frame}
    \frametitle{Generative Reactor Design Optimization}
    Generative design is an \textbf{iterative design exploration process} \cite{autodesk_autodesk_2020}. 
    \begin{itemize}
        \item Designers provide design goals and constraints to the generative design 
        software
        \item Software explores all the possible permutations of a solution, quickly generating 
        design alternatives
    \end{itemize}
    \vspace{0.2cm}
    The generative design software does not replace the human reactor designer but 
    \textbf{shifts the human designer's focus} away from conjecturing suitable geometries 
    to \textbf{defining the design criteria of optimal designs}.
    \\
    \textbf{Evolutionary algorithms} can be used to \textbf{drive generative reactor 
    design optimization} to promptly explore the large design space to find global optimal 
    designs. 
  \end{frame}

    \begin{frame}
    \frametitle{Evolutionary Algorithms for Reactor Generative Design}
    Evolutionary algorithms imitate natural selection to evolve solutions. 
    \begin{minipage}[c]{0.6\textwidth}
            Evolutionary Algorithm Benefits 
            \begin{itemize}
                \item Successful at finding multi-objective problems' global optimum 
                \item Easily parallelized 
            \end{itemize}
        \end{minipage}\hfill
        \begin{minipage}[c]{0.4\textwidth}
            \begin{figure}
                \includegraphics[width=\linewidth]{figures/ea-flow.png} 
                \caption{Evolutionary algorithm flow \cite{renner_genetic_2003}. }
              \end{figure}
        \end{minipage}
    \end{frame}

  \begin{frame}
    \frametitle{Evolutionary Algorithm Optimization for Reactor Generative Design}
    \begin{minipage}[c]{0.6\textwidth}
        \begin{itemize}
            \item We can leverage evolutionary algorithm optimization to 
            explore the large design space enabled by 3D printing to find global 
            optimal designs
            \item Evolutionary algorithms have proven successful in optimizing 
            multi-objective problems as they can find solutions at the global 
            optimum and can be run in parallel
            \item Evolutionary algorithms imitate natural selection to evolve solutions 
            \end{itemize}
  \end{minipage}\hfill
  \begin{minipage}[c]{0.4\textwidth}
    \centering
    \begin{figure}
      \includegraphics[width=0.8\linewidth]{figures/ea-flow.png} 
      \caption{Evolutionary algorithm flow \cite{renner_genetic_2003}. }
    \end{figure}
  \end{minipage}
  \end{frame}

\subsection{Objectives: AHTR Optimization for Non-Conventional Designs}
\begin{frame}
    \frametitle{Research Objectives: AHTR optimization for non-conventional designs}
    \begin{block}{Technical Gap}
      \begin{itemize}
        \item Optimization tools for generating new reactor designs enabled by
        3D printing do not exist
        \item Reactor optimization for non-conventional geometries and parameters 
        has not been done 
      \end{itemize}
    \end{block}
    \begin{block}{Research Objectives: AHTR optimization for non-conventional designs}
        \begin{itemize}
            \item Develop an open-source tool that enables generative reactor design 
            optimization with evolutionary algorithms for non-conventional reactor
            geometries and fuel distributions
            \item Demonstrate successful application of the optimization tool 
            for single and multi-objective AHTR optimization
        \end{itemize}
    \end{block}
  \end{frame}

\subsection{Summary}
\begin{frame}
    \frametitle{Research Objectives: Summary}
    \textbf{Additive manufacturing has the potential to radically transform reactor 
    design.}
    \begin{block}{Research Objectives: AHTR Model Development}
        \begin{itemize}
            \item Participate in the OECD-NEA's FHR Benchmark to further our understanding 
            of the AHTR design's complexities through neutronics modeling with 
            OpenMC \cite{romano_openmc:_2015}
            \item AHTR Moltres \cite{lindsay_moltres_2017} temperature model
        \end{itemize}
    \end{block}

    \begin{block}{Research Objectives: AHTR optimization for non-conventional designs}
        \begin{itemize}
            \item Develop an open-source tool that enables generative reactor design 
            optimization with evolutionary algorithms for non-conventional reactor
            geometries and fuel distributions
            \item Demonstrate successful application of the optimization tool 
            for single and multi-objective AHTR optimization
        \end{itemize}
    \end{block}
\end{frame}