\begin{frame}
    \frametitle{3D Printing a Nuclear Reactor?}
    \begin{block}{Impact of Additive Manufacturing Technology Advancements on 
        Reactor Design Optimization}
        \begin{itemize}
            \item Leveraging additive manufacturing enables us to surpass classical 
            manufacturing constraints and optimize for arbitrary geometries and parameters 
            such as non-uniform channel shapes, and inhomogeneous fuel distribution 
            throughout the core
            \item Wide-spread adoption of additive manufacturing methods in the nuclear industry 
            could drastically reduce reactor fabrication costs and deployment timelines 
            and improve reactor safety
          \end{itemize}
    \end{block}
  \end{frame}

  \begin{frame}
    \frametitle{Evolutionary Algorithm Optimization}
    \begin{block}{Evolutionary Algorithms for Reactor Design Optimization}
    \begin{minipage}[c]{0.6\textwidth}
        \begin{itemize}
            \item We can leverage evolutionary algorithm optimization to 
            explore the large design space enabled by 3D printing to find global 
            optimal designs
            \item Evolutionary algorithms have proven successful in optimizing 
            multi-objective problems as they can find solutions at the global 
            optimum and can be run in parallel
            \item Evolutionary algorithms imitate natural selection to evolve solutions 
            \end{itemize}
  \end{minipage}\hfill
  \begin{minipage}[c]{0.4\textwidth}
    \centering
    \begin{figure}
      \includegraphics[width=0.8\linewidth]{figures/ea-flow.png} 
      \caption{Evolutionary algorithm flow \cite{renner_genetic_2003}. }
    \end{figure}
  \end{minipage}
  \end{block}
  \end{frame}
    