\subsection{ROLLO: Reactor evOLutionary aLgorithm Optimizer}
\begin{frame}
    \frametitle{ROLLO: Reactor evOLutionary aLgorithm Optimizer}
    \begin{figure}
        \includegraphics[width=0.7\linewidth]{figures/rollo-logo.png} 
        \caption{ROLLO (Reactor evOLutionary aLgorithm Optimizer) logo.}
    \end{figure}
    \begin{itemize}
        \item ROLLO (Reactor evOLutionary aLgorithm Optimizer) is a Python package 
        that applies evolutionary algorithms to optimize nuclear reactor design
        \item ROLLO provides a general genetic algorithm framework, sets up 
        parallelization for the user, and promotes usability with an input file that
        only exposes mandatory parameters
        \item Designed to be: effective, flexible, open-source, parallel, reproducible
    \end{itemize}
\end{frame}

\begin{frame}
    \frametitle{How does ROLLO work?}
    \begin{columns}
        \begin{column}{0.7\textwidth}
            
            \vspace{-0.5cm}
            ROLLO Flow 
            \begin{itemize}
                \item Reads and validates the JSON input file
                \item Initializes the \acrfull{DEAP} genetic algorithm hyperparameters
                \item Runs the genetic algorithm  
                \item During the run, nuclear modeling software evaluates each individual 
                reactor model's fitness
            \end{itemize}

            \vspace{0.2cm}
            For each ROLLO optimization simulation, one must \textbf{balance convergence and 
            computational cost}. 

            \vspace{0.2cm}
            ROLLO's purpose is to help the human reactor designer narrow down reactor design 
            search space. The reactor designer uses the \textbf{computational power 
            available}, to \textbf{narrow down the search space} as much as possible.
        \end{column}
        \begin{column}{0.3\textwidth}
            \begin{figure}
                \includegraphics[width=0.8\linewidth]{figures/rollo-flow2.png} 
                \caption{ROLLO's Genetic Algorithm Flow.}
            \end{figure}
        \end{column}
    \end{columns}
\end{frame}
